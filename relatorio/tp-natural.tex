\documentclass[12pt]{article}

\usepackage[portuguese,linesnumbered,ruled,vlined]{algorithm2e}
\usepackage{sbc-template}
\usepackage{graphicx,url,amsfonts}

\usepackage[brazil]{babel}
\usepackage[utf8]{inputenc}
\usepackage{hyperref}

\pagenumbering{arabic}

\sloppy

\title{I Trabalho Prático de Programação Natural\\Cubo Mágico}

\author{Gabriel de Biasi\inst{1}}

\address{
Departamento de Ciência da Computação\\
Universidade Federal de Minas Gerais\\
Av. Antônio Carlos, 6627 -- Pampulha -- Belo Horizonte -- MG
\email{biasi@dcc.ufmg.br}
}

\begin{document}

\maketitle

\section{Descrição do Problema}
  O problema do \textbf{ZicaZeroAnelDual}.

\section{Objetivo}
  Nesta seção, \cite{cormen2009}.

\section{Estado da Arte (baseline)}
  lala

\section{Solução Proposta}
  lala

\section{Conclusão}
  Neste trabalho.

\bibliographystyle{sbc}
\bibliography{ref}

\end{document}
